\chapter{Observações finais}

Existem diversos métodos disponíveis para modelagem geológica automática e avaliação de incerteza em modelos geológicos, cada um deles com suas características e particularidades. Não obstante, a modelagem geológica implícita com funções distância assinaladas tem se tornado extremamente popular ao longo da última década, ganhando espaço nos principais softwares de mineração. Por esse motivo, essa tese se propôs a desenvolver e investigar métodos para avaliação de incerteza em modelos geológicos multi-categóricos usando funções distância assinaladas. Tendo em vista que o modelo geológico pode ser o responsável por incertezas cruciais em um projeto mineiro a incerteza associada ao modelo geológico não deve ser negligenciada.

Nesse sentido, foram propostos três métodos: um baseado em estimativa global e dois baseados em simulações não condicionais no interior de uma zona de incerteza. A tese apresentou os métodos a partir de um conceito de prova no conhecido banco de dados \textit{Swiss Jura}, e posteriormente, conduziu um estudo de caso em um banco de dados, sintético ou real, a fim de verificar a competência dos métodos propostos. Os resultados foram discutidos e comparados com métodos concorrentes. As vantagens, desvantagens e aplicabilidade de cada método foram apontadas.

\section{Sumário das contribuições}

\begin{itemize}
    \item \textbf{Parametrização do suporte do \textit{kernel}}: o método avalia a incerteza associada ao modelo geológico "simulando" diferentes interpretações para cada uma das litologias, contraindo ou expandindo seu volume ou área. O método é baseado em interpolação global e por isso gera formas geológicas suaves e realistas. A implementação do método ainda necessita de aprimoramentos e otimização de \textit{software}.
    \item \textbf{Avaliação de incerteza usando funções distância assinaladas e campos de probabilidades}: o método é bastante simples e cômodo já que dissocia a simulação dos contatos do cálculo das probabilidades locais. As probabilidades locais precisam ser calculadas para a confecção dos modelos determinísticos, sendo assim, simular os contatos requer apenas a definição da zona de incerteza e simulações não condicionais Gaussianas. Em contrapartida, há limitações, já que a zona de incerteza e as simulações não condicionais devem ser os mesmos para todas as diferentes litologias. A combinação dos parâmetros \textit{omega} e alcance do variograma das simulações pode gerar estruturas descontínuas. 
    \item \textbf{simulação hierárquica dos contatos}: a simulação hierárquica dos contatos, assim como a avaliação de incerteza usando funções distância assinaladas e campos de probabilidades, também é baseada em simulações não condicionais no interior de uma zona de incerteza. Porém, é um método mais laborioso, já que exige que definição de grupos, a definição de zonas de incertezas para cada um dos grupos e posterior simulações Gaussianas não condicionais para cada um dos grupos. Todavia, isso também faz com que o método seja mais flexível já que é possível aplicar diferentes zonas de incertezas e diferentes simulações para cada um dos grupos. O método é robusto no sentido de gerar estruturas realistas já que a comparação entre distâncias interpoladas e valor simulado bloco a bloco gera limites contínuos. A definição dos grupos traz subjetividade ao método. Mesmo a implementação automática, apesar de poupar tempo de trabalho do geomodelador, depende de um proto-modelo: diferentes proto-modelos geram diferentes regras hierárquicas.
    \item \textbf{\textit{Software}}: foi desenvolvida uma suite de \textit{plugins} para o software geostatístico \textit{AR2GeMS} que implementa todas as ferramentas necessárias para a modelagem geológica com distâncias assinaladas ou indicadores incluindo: cálculo das distâncias ou indicadores; criação de modelos determinísticos a partir das distâncias assinaladas, dos indicadores ou usando suporte de vetores de máquina (SVM); avaliação de incerteza dos modelos a partir da parametrização do \textit{kernel} e usando campos de probabilidades; ferramentas auxiliares para criação automática de \textit{grids} e exportação para visualização em softwares especializados. O método simulação hierárquica de contatos, por ser bastante dependente de ferramentas já implementadas na biblioteca \textit{GSLib}, foi implementado a partir de \textit{jupyter notebooks} e módulos auxiliares em Python disponibilizados em um repositório git.
\end{itemize}

\section{Sugestões para trabalhos futuros}

Dada a limitação de tempo para conclusão da tese algumas ideias e aprimoramentos foram deixados para trabalhos futuros. 

 \begin{itemize}
    \item Definir uma metodologia para determinar os variogramas das simulações não condicionais nos métodos dos campos potenciais e simulação hierárquica; 
    \item Definir uma metodologia objetiva para calibrar a zona de incerteza nos métodos dos campos potenciais e simulação hierárquica. A zona de incerteza não necessariamente deve ser simétrica ao redor dos contatos. Não devem haver amostras no interior da zona de incerteza;
    \item Aprimorar a parametrização do suporte do \textit{kernel} para que a sensibilidade ao parâmetro fmin seja reduzida;
    \item Conduzir uma análise de risco mais robusta com simulação de teores e do modelo geológico em um banco de dados real comparando os resultados com os teores krigados em um modelo geológico determinístico; 
    \item Desenvolver uma metodologia para validação dos modelos geológicos. Usar redes neurais treinadas em seções de modelos determinísticos para identificar estruturas contínuas e realistas é uma avenida de investigação;
    \item Desenvolver metodologias, ou adaptações de metodologias existentes, para modelar estruturas geológicas especĩficas (dobras, diques, falhas, lentes) e avaliar sua incerteza.
 \end{itemize}