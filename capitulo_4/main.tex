\chapter{Conclusão}

Os métodos geoestatísticos requerem a delimitação prévia de domínios estacionários para modelagem do depósito mineral. Na indústria de mineração, modelos geológicos determinísticos são amplamente utilizados. No entanto, esses modelos não permitem a avaliação da incerteza em volumes e toneladas. A geometria dos domínios geológicos afeta diretamente a avaliação dos recursos minerais, pois determina os volumes de minério e estéril disponíveis e pode ser responsável pelas maiores incertezas em um projeto de mineração. Assim, o impacto é significativo na estimativa da tonelagem de minério, levando a estimativas enviesadas, erros no planejamento de lavra, custos inesperados na operação e desvios nas receitas esperadas \cite{srivastava2005probabilistic}.



melhor forma de validação 

melhor forma de calibrar as zonas de incertezas, não pode ter amostra dentro

pfields pode criar modelos descontinuos e não realistas boundary simulation não!

forma não subjetiva de definir o variograma das simulações não condicionais.

diferenças essenciais entre os metodos. zona de incerteza, contao certinho, etc...

mostrar imagens de entropia pfields incert zona simetrica em tornop dos contatos de todas as cats hier nao 

\section{Sugestões para trabalhos futuros}

